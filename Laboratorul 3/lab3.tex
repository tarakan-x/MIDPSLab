\documentclass[11pt]{article}
\usepackage{graphicx}
\usepackage{listings}
\usepackage{hyperref}



\begin{document}
\begin{titlepage}

\begin{center}

\textsc{\large Universitatea Tehinica a Moldovei}\\[0.5cm]
\textsc{\large Facultatea Calculatoare,Informatica si Microelectronica}\\[0.5cm]
\vspace{25 mm}
\textsc{\large MIDPS}\\[0.5cm]
\textbf{\large Lucrare de laborator \#3}\\[0.5cm]
 \vspace{45 mm}

      \begin{minipage}{0.4\textwidth}
      \begin{flushleft} \large
      \emph{Autor:}\\
      Chifa Vladislav
      \end{flushleft}
      \end{minipage}
      ~
      \begin{minipage}{0.4\textwidth}
      \begin{flushright} \large
      \emph{lector asistent:} \\
      Irina Cojanu \\ 
      \emph{lector superior:}\\
     Cojocaru Svetlana
      \end{flushright}
      \end{minipage}\\[4cm]

      \vspace{10 mm}
   
      \textbf{\large Chisinau 2016}
     \newpage
       

\end{center}
\end{titlepage}

\section*{Lucrare de Laborator \#3}

\section{Scopul lucrarii de laborator}
Realizeaza un simplu GUI calculator care suporta urmatoare functii: +, -, /, *, putere, radical, InversareSemn(+/-), operatii cu numere zecimale.

\section{Obiective}
\begin{enumerate}
\item Realizeaza un simplu GUI Calculator.
\item Operatiile simple: +,-,*,/,putere,radical,InversareSemn(+/-),operatii cu numere zecimale.
\item Divizare proiectului in doua module - Interfata grafica(Modul GUI) si Modulul de baza(Core Module).
\end{enumerate}

\section{Listingul Programului.}
\lstdefinestyle{sharpc}{language=[Sharp]C}

\lstset{style=sharpc}
\begin{lstlisting}
using System;
using System.Collections.Generic;
using System.ComponentModel;
using System.Data;
using System.Drawing;
using System.Linq;
using System.Text;
using System.Threading.Tasks;
using System.Threading;
using System.Threading.Tasks;
using System.Globalization;
using System.Windows.Forms;

namespace CalculatorV2
{
     public partial class Form1 : Form
     {

          Double resultVal = 0;
          String opPerformed = "";
          bool isopPerformed = false;
          bool s = true;


          public Form1()
          {
               InitializeComponent();
               Thread.CurrentThread.CurrentCulture =
                Thread.CurrentThread.CurrentUICulture =
                 new CultureInfo("en-US");
          }


          private void button1_Click(object sender, EventArgs e)
          {
           if ((textBox1_Result.Text == "0") || (isopPerformed))
          textBox1_Result.Clear();

           isopPerformed = false;
           Button button = (Button)sender;
           if (button.Text == ".")
           {
            if (!textBox1_Result.Text.Contains("."))
            textBox1_Result.Text = textBox1_Result.Text
               + button.Text;
               }
               else
              textBox1_Result.Text = textBox1_Result.Text +
               button.Text;
          }

          private void func_Click(object sender, EventArgs e)
          {
               Button button = (Button)sender;
               if (resultVal != 0)
               {
                    if (button.Text == "Sqrt")
      textBox1_Result.Text =
   Math.Sqrt(Double.Parse(textBox1_Result.Text)).ToString();
                    if (button.Text == "Log")
    textBox1_Result.Text =
     Math.Log(Double.Parse(textBox1_Result.Text)).ToString();
                    if (button.Text == "^2")
   textBox1_Result.Text =
    Math.Pow(Double.Parse(textBox1_Result.Text), 2).ToString();




                    equal.PerformClick();
                    opPerformed = button.Text;
         // lbCurentOp.Text = resultVal + " " + opPerformed;
                    isopPerformed = true;
               }
               else if (button.Text == "Sqrt")
               {
       textBox1_Result.Text =
      Math.Sqrt(Double.Parse(textBox1_Result.Text)).ToString();
    resultVal = 
    Math.Sqrt(Double.Parse(textBox1_Result.Text));
               }
               else if (button.Text == "Log")
               {
     textBox1_Result.Text =
      Math.Log(Double.Parse(textBox1_Result.Text)).ToString();
  resultVal = Math.Log(Double.Parse(textBox1_Result.Text));
               }
  else if (button.Text == "^2")
               {
        textBox1_Result.Text =
    
     Math.Pow(Double.Parse(textBox1_Result.Text), 2).ToString();
           resultVal = Math.Pow(Double.Parse
           (textBox1_Result.Text), 2);
               }
               else
               {
                    opPerformed = button.Text;
                    resultVal = Double.Parse(textBox1_Result.Text);
                    opPerformed = button.Text;
                   //lbCurentOp.Text = resultVal + 
                   " " + opPerformed;
                    isopPerformed = true;

               }
              //lbCurentOp.Focus();
          }

      private void deletCar_Click(object sender, EventArgs e)
          {
            int length = textBox1_Result.Text.Length - 1;
            string text = textBox1_Result.Text;
            textBox1_Result.Clear();
          for (int i = 0; i < length; i++)
               {
             textBox1_Result.Text = textBox1_Result.Text 
             + text[i];
               }
          }

          private void clear_Click(object sender, EventArgs e)
          {
               textBox1_Result.Text = "0";
               resultVal = 0;
          }

        
    private void button10_Click(object sender, EventArgs e)
          {
              if (s == true)
            {
        textBox1_Result.Text =
         "-" + textBox1_Result.Text;
                s = false;
            }
            else
            {
         textBox1_Result.Text =
          textBox1_Result.Text.Replace("-", "");
                s = true;
            }
          }

          private void equal_Click(object sender, EventArgs e)
          {
               switch (opPerformed)
               {


        case "+":
     textBox1_Result.Text = 
     (resultVal + Double.Parse(textBox1_Result.Text)).ToString();
                         break;
                    case "-":
                         textBox1_Result.Text = 
                         (resultVal - Double.Parse(textBox1_Result.Text)).ToString();
                         break;
                    case "*":
                         textBox1_Result.Text = 
                         (resultVal * Double.Parse(textBox1_Result.Text)).ToString();
                         break;
                    case "/":
                         textBox1_Result.Text =
                          (resultVal / Double.Parse(textBox1_Result.Text)).ToString();
                         break;
                default:
                     break;
             }

           resultVal = 
           Double.Parse(textBox1_Result.Text);
               //lbCurentOp.Text = "";
          }

       

          private void operator_Click(object sender, EventArgs e)
          {
               Button button = (Button)sender;
               opPerformed = button.Text;
               resultVal = Double.Parse(textBox1_Result.Text);
               isopPerformed = true;
          }

          private void zecimal_Click(object sender, EventArgs e)
          {
               Button button = (Button)sender;
               if (button.Text == ".")
               {
               if (!textBox1_Result.Text.Contains("."))
                 textBox1_Result.Text = textBox1_Result.Text
                  + button.Text;
               }
               else
        textBox1_Result.Text = textBox1_Result.Text
         + button.Text;
          }

          private void Form1_KeyPress(object sender, 
          KeyPressEventArgs e)
          {
               switch (e.KeyChar)
               {
                    case "0":
                         zero.PerformClick();
                         break;
                    case "1":
                         one.PerformClick();
                         break;
                    case "2":
                         two.PerformClick();
                         break;
                    case "3":
                         three.PerformClick();
                         break;
                    case "4":
                         four.PerformClick();
                         break;
                    case "5":
                         five.PerformClick();
                         break;
                    case "6":
                         six.PerformClick();
                         break;
                    case "7":
                         seven.PerformClick();
                         break;
                    case "8":
                         eight.PerformClick();
                         break;
                    case "9":
                         nine.PerformClick();
                         break;
                    case "+":
                         add.PerformClick();
                         break;
                    case "-":
                         sub.PerformClick();
                         break;
                    case "*":
                         mul.PerformClick();
                         break;
                    case "/":
                         div.PerformClick();
                         break;
                    case "=":
                         equal.PerformClick();
                         break;
                    default:
                         break;
               }
          }
     }
}
\end{lstlisting}
\begin{center}


\includegraphics[scale=1]{C:/Users/bossa/OneDrive/Imagini/Capturi/ffffffff.png} 
\end{center}
 
\section{Concluzie}
Efectuind aceasta lucrare de la borator am folosit IDE Visual Studio unde am realizat un simplu GUI  calculator  care suporta functiile +,-,/,*, radacina patrata dintrun numar , ridicarea la putere si schimbarea semnului in limbajul de programare C Sharp.

\section{Bibliografie}
\url{https://www.youtube.com/watch?v=iJqB6UsM-hs&nohtml5=False}

\end{document}