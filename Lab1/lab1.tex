\documentclass[11pt]{article}
\usepackage{graphicx}



\begin{document}


\begin{titlepage}
\begin{center}


\textsc{\large Universitatea Tehinica a Moldovei}\\[0.5cm]
\textsc{\large Facultatea Calculatoare,Informatica si Microelectronica}\\[0.5cm]
\vspace{25 mm}
\textsc{\large Medii Interactive de Dezvoltare a Produselor Soft}\\[0.5cm]
\textbf{\large Lucrare de laborator \#1}\\[0.5cm]
 \vspace{45 mm}

      \begin{minipage}{0.4\textwidth}
      \begin{flushleft} \large
      \emph{Autor:}\\
      Chifa Vladislav
      \end{flushleft}
      \end{minipage}
      ~
      \begin{minipage}{0.4\textwidth}
      \begin{flushright} \large
      \emph{lector asistent:} \\
      Irina Cojanu \\ 
      \emph{lector superior:}\\
      Svetlana Cojocaru 
      \end{flushright}
      \end{minipage}\\[4cm]

      \vspace{10 mm}
   
      \textbf{\large Chisinau 2016}
     \newpage
       

\end{center}
\end{titlepage}


\begin{center}
   \section*{Lucrare de laborator\#1}
   \textbf{\large Mediu Integrat C++ Builder}\\[0.5cm]
\end{center}

\section{Obiective}
\begin{enumerate}
\item Insusirea modului de utilizare a celor mai importante componente ale mediului integrat C++ BUILDER . Realizarea unui program simplu care utilizează componente de tip TButton, TEdit, Tlabel, RadioButton  etc.
\item Insusirea modului de utilizare a componentei VCL TTimer. Insusirea modului de utilizare a funcsiilor de lucru cu timpul sistem. Realizarea unor aplicatii de gestionare a resursei timp. 
\item Insusirea modului de utilizare a componentelor VCL  TPaintBox si TPanel. Insusirea modului de utilizare a principalelor functii grafice ale mediului C++BUILDER . Realizarea unor elemente pentru  afisarea grafica a informatiei (diagrama si bargraf).  
 
\end{enumerate}
\section{Notiuni Teoretice}
\textit{Utilizarea componentei TTimer}\\[0.3 cm]
 Componenta Ttimer se gaseste in Componente Palatte(pagina System).Obiectul de acest tip permite ecxecutia in cadrul aplicatiei a unor functii la intervale specificate .In context Windows obiactul TTimer lanseaza catre aplicatie mesaje la intervale prestabile.\\[0.3 cm]
 
 \textit{Utilizarea componentei TpaintBox}\\[0.3 cm]
 Componeta TPaintBox se gaseste in Copponent Panel(pagina System).Obieactul de acest tip furnizeaza o componenta de tip TCanvas care permite desenarea in interiorul unui dreptunghi,prevenind depasirea marginilor acestuia.\\[0.3 cm]
 
 \textit{Utilizarea componentei Tpanel}\\[0.3 cm]
 Componenta TPanel se gaseste in Componet Pallete (pagina System).
 Obiectul de acest tip poate fi utilizat pentru desenarea daca pe el se amplaseaza o componenta TPaintBox.\\[0.3 cm]
 
 
   \section{Realizarea sarcinii.}
 
 Declaram variabila nr de tip intreg, cu valoarea initiala egala cu zero. Aceasta variabila va fi incrementata sau decrementata in dependenta de butonul apasat pe forma. Am definit un sir de caractere s cu dimensiunea maxima egala cu 10.
 \begin{center}
 \texttt{ int nr = 0;\\char s[10];}
 \end{center}
 Am scris o functie  iValoare() care realizeaza convertirea numarului care trebuie afisat din tipul intreg in sir de caractere. Pentru convertire am utilizat functia itoa care are 3 parametri : variabila de tip intreg care trebuie convertita, variabila sir in care se inscrie sirul si baza numarului care urmeaza sa fie convertit. Dupa convertire, textul din sirul s se inscrie in componenta Edit1 din Form2.\\
 
\begin{center}
\texttt{ void inscrieValoare\{ \\
  itoa(nr, s, 10);\\
  Form2->Edit1->Text = s;\}\\}
\end{center}
Am definit evenimentul pentru apasarea butonului Close. La apasarea butonului se realizeaza iesirea din executia programului, prin apelarea functiei Close().\\

\begin{center}
\texttt{void fastcall TForm2::Button3Click(TObject *Sender)\{\\ Close();\\
\}\\}
\end{center}

Definim evenimentul pentru butonul Up. La apasarea butonului se realizeaza incrementarea variabilei nr apoi se apeleaza functia iValoare() pentru convertirea noii valori a lui nr in sir de caractere si afisarea lui.
\begin{center}
\texttt{ ++nr;\\ iValoare();\\}
\end{center}

Definim evenimentul pentru butonul Down. La apasarea butonului se realizeaza decrementarea variabilei nr apoi se apeleaza functia iValoare() pentru convertirea noii valori a lui nr in sir de caractere si afisarea lui.\\
   \begin{center}
   \texttt{--nr;\\ iValoare();\\}
   \end{center}
   Am definit un Timer2 si un Panel2 unde cu ajutorul functiei Time();   afisam ora curenta.
   \begin{center}
   \texttt{void fastcall TForm2::Timer2Timer(TObject *Sender)\{ \\
     Panel2->Caption=Time(); \\ \}
     \\}
   \end{center}
   
   Am definit evenimentul pentru butonul Bargraf care la apasarea butonului se deseneaza o diagrama intr-un PaintBox2.\\
   
   \begin{center}
   \texttt{void fastcall TForm2::Button9Click(TObject *Sender)\{\\
 PaintBox2->Canvas->MoveTo(10,50);\\
PaintBox2->Canvas->LineTo(10,40);\\
PaintBox2->Canvas->LineTo(50,40);\\
PaintBox2->Canvas->LineTo(50,50);\\
PaintBox2->Canvas->MoveTo(50,40);\\
PaintBox2->Canvas->LineTo(50,30);\\
PaintBox2->Canvas->LineTo(90,30);\\
PaintBox2->Canvas->LineTo(90,50);\\
PaintBox2->Canvas->MoveTo(90,30);\\
PaintBox2->Canvas->LineTo(90,20);\\
PaintBox2->Canvas->LineTo(130,20);\\
PaintBox2->Canvas->LineTo(130,50);\\
PaintBox2->Canvas->MoveTo(130,20);\\
PaintBox2->Canvas->LineTo(130,10);\\
PaintBox2->Canvas->LineTo(170,10);\\
PaintBox2->Canvas->LineTo(170,50);\\
PaintBox2->Canvas->MoveTo(170,10);\\
PaintBox2->Canvas->LineTo(170,0);\\
PaintBox2->Canvas->LineTo(220,0);\\
PaintBox2->Canvas->LineTo(220,50);\\ \}\\}
   \end{center}
   
   Am definit evenimetul pentru RadioButton-uri la care leam atribuit functionalu de a schimba Font-ul , marimea si silul  unui text .\\
   \begin{center}
   •\texttt{void fastcall TForm2::RadioButton1Click(TObject *Sender)\{ \\
Label1->Font->Name="Arial";\\ \}}
   \end{center}
   
   \section{Rezultatul}
 \begin{center}
\includegraphics[width= 13 cm]{OneDrive/Imagini/"Capturi de ecran"/Lab1.png} 
\end{center}


\section{Concluzie}
Efectuind aceasta lucrare de laborator am facut cunostinte cu mediul de integrat C++ Builde r, am insusit  modil de utilizare a celor mai importante  componente ale lui , am folosit componente de tip RadioButton, Panel, VCL Timer,PaintBox etc.

\end{document}
 